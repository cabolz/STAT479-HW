% Options for packages loaded elsewhere
\PassOptionsToPackage{unicode}{hyperref}
\PassOptionsToPackage{hyphens}{url}
%
\documentclass[
]{article}
\usepackage{lmodern}
\usepackage{amssymb,amsmath}
\usepackage{ifxetex,ifluatex}
\ifnum 0\ifxetex 1\fi\ifluatex 1\fi=0 % if pdftex
  \usepackage[T1]{fontenc}
  \usepackage[utf8]{inputenc}
  \usepackage{textcomp} % provide euro and other symbols
\else % if luatex or xetex
  \usepackage{unicode-math}
  \defaultfontfeatures{Scale=MatchLowercase}
  \defaultfontfeatures[\rmfamily]{Ligatures=TeX,Scale=1}
\fi
% Use upquote if available, for straight quotes in verbatim environments
\IfFileExists{upquote.sty}{\usepackage{upquote}}{}
\IfFileExists{microtype.sty}{% use microtype if available
  \usepackage[]{microtype}
  \UseMicrotypeSet[protrusion]{basicmath} % disable protrusion for tt fonts
}{}
\makeatletter
\@ifundefined{KOMAClassName}{% if non-KOMA class
  \IfFileExists{parskip.sty}{%
    \usepackage{parskip}
  }{% else
    \setlength{\parindent}{0pt}
    \setlength{\parskip}{6pt plus 2pt minus 1pt}}
}{% if KOMA class
  \KOMAoptions{parskip=half}}
\makeatother
\usepackage{xcolor}
\IfFileExists{xurl.sty}{\usepackage{xurl}}{} % add URL line breaks if available
\IfFileExists{bookmark.sty}{\usepackage{bookmark}}{\usepackage{hyperref}}
\hypersetup{
  pdftitle={479 HW 1},
  pdfauthor={Caitlin Bolz},
  hidelinks,
  pdfcreator={LaTeX via pandoc}}
\urlstyle{same} % disable monospaced font for URLs
\usepackage[margin=1in]{geometry}
\usepackage{color}
\usepackage{fancyvrb}
\newcommand{\VerbBar}{|}
\newcommand{\VERB}{\Verb[commandchars=\\\{\}]}
\DefineVerbatimEnvironment{Highlighting}{Verbatim}{commandchars=\\\{\}}
% Add ',fontsize=\small' for more characters per line
\usepackage{framed}
\definecolor{shadecolor}{RGB}{248,248,248}
\newenvironment{Shaded}{\begin{snugshade}}{\end{snugshade}}
\newcommand{\AlertTok}[1]{\textcolor[rgb]{0.94,0.16,0.16}{#1}}
\newcommand{\AnnotationTok}[1]{\textcolor[rgb]{0.56,0.35,0.01}{\textbf{\textit{#1}}}}
\newcommand{\AttributeTok}[1]{\textcolor[rgb]{0.77,0.63,0.00}{#1}}
\newcommand{\BaseNTok}[1]{\textcolor[rgb]{0.00,0.00,0.81}{#1}}
\newcommand{\BuiltInTok}[1]{#1}
\newcommand{\CharTok}[1]{\textcolor[rgb]{0.31,0.60,0.02}{#1}}
\newcommand{\CommentTok}[1]{\textcolor[rgb]{0.56,0.35,0.01}{\textit{#1}}}
\newcommand{\CommentVarTok}[1]{\textcolor[rgb]{0.56,0.35,0.01}{\textbf{\textit{#1}}}}
\newcommand{\ConstantTok}[1]{\textcolor[rgb]{0.00,0.00,0.00}{#1}}
\newcommand{\ControlFlowTok}[1]{\textcolor[rgb]{0.13,0.29,0.53}{\textbf{#1}}}
\newcommand{\DataTypeTok}[1]{\textcolor[rgb]{0.13,0.29,0.53}{#1}}
\newcommand{\DecValTok}[1]{\textcolor[rgb]{0.00,0.00,0.81}{#1}}
\newcommand{\DocumentationTok}[1]{\textcolor[rgb]{0.56,0.35,0.01}{\textbf{\textit{#1}}}}
\newcommand{\ErrorTok}[1]{\textcolor[rgb]{0.64,0.00,0.00}{\textbf{#1}}}
\newcommand{\ExtensionTok}[1]{#1}
\newcommand{\FloatTok}[1]{\textcolor[rgb]{0.00,0.00,0.81}{#1}}
\newcommand{\FunctionTok}[1]{\textcolor[rgb]{0.00,0.00,0.00}{#1}}
\newcommand{\ImportTok}[1]{#1}
\newcommand{\InformationTok}[1]{\textcolor[rgb]{0.56,0.35,0.01}{\textbf{\textit{#1}}}}
\newcommand{\KeywordTok}[1]{\textcolor[rgb]{0.13,0.29,0.53}{\textbf{#1}}}
\newcommand{\NormalTok}[1]{#1}
\newcommand{\OperatorTok}[1]{\textcolor[rgb]{0.81,0.36,0.00}{\textbf{#1}}}
\newcommand{\OtherTok}[1]{\textcolor[rgb]{0.56,0.35,0.01}{#1}}
\newcommand{\PreprocessorTok}[1]{\textcolor[rgb]{0.56,0.35,0.01}{\textit{#1}}}
\newcommand{\RegionMarkerTok}[1]{#1}
\newcommand{\SpecialCharTok}[1]{\textcolor[rgb]{0.00,0.00,0.00}{#1}}
\newcommand{\SpecialStringTok}[1]{\textcolor[rgb]{0.31,0.60,0.02}{#1}}
\newcommand{\StringTok}[1]{\textcolor[rgb]{0.31,0.60,0.02}{#1}}
\newcommand{\VariableTok}[1]{\textcolor[rgb]{0.00,0.00,0.00}{#1}}
\newcommand{\VerbatimStringTok}[1]{\textcolor[rgb]{0.31,0.60,0.02}{#1}}
\newcommand{\WarningTok}[1]{\textcolor[rgb]{0.56,0.35,0.01}{\textbf{\textit{#1}}}}
\usepackage{graphicx,grffile}
\makeatletter
\def\maxwidth{\ifdim\Gin@nat@width>\linewidth\linewidth\else\Gin@nat@width\fi}
\def\maxheight{\ifdim\Gin@nat@height>\textheight\textheight\else\Gin@nat@height\fi}
\makeatother
% Scale images if necessary, so that they will not overflow the page
% margins by default, and it is still possible to overwrite the defaults
% using explicit options in \includegraphics[width, height, ...]{}
\setkeys{Gin}{width=\maxwidth,height=\maxheight,keepaspectratio}
% Set default figure placement to htbp
\makeatletter
\def\fps@figure{htbp}
\makeatother
\setlength{\emergencystretch}{3em} % prevent overfull lines
\providecommand{\tightlist}{%
  \setlength{\itemsep}{0pt}\setlength{\parskip}{0pt}}
\setcounter{secnumdepth}{-\maxdimen} % remove section numbering
\usepackage {hyperref}
\hypersetup {colorlinks = true, linkcolor = red, urlcolor = red}

\title{479 HW 1}
\author{Caitlin Bolz}
\date{2/3/2021}

\begin{document}
\maketitle

\hypertarget{instructions}{%
\subsection{Instructions}\label{instructions}}

\begin{enumerate}
\def\labelenumi{\arabic{enumi}.}
\tightlist
\item
  Submit your solutions to the exercises below before \textbf{February 5
  at 11:59pm CST}.
\item
  Prepare your solutions as Rmarkdown documents. For problems in
  vega-lite, copy your code into the \texttt{.Rmd}, making sure to
  enclose them within \texttt{js} blocks,
\end{enumerate}

\begin{verbatim}
````js
// your code here
````
\end{verbatim}

so that the syntax is properly highlighted, e.g.,

\begin{Shaded}
\begin{Highlighting}[]
\CommentTok{// your code here}
\end{Highlighting}
\end{Shaded}

\begin{enumerate}
\def\labelenumi{\arabic{enumi}.}
\setcounter{enumi}{2}
\tightlist
\item
  For vega-lite problems, include a screenshot of your result.
\item
  We give example figures below to guide your work. However, view these
  only as suggestions -- we will keep an eye out for improvements over
  our plots.
\item
  Include two files in your submission, (a) a pdf of the compiled
  \texttt{.Rmd} file and (b) the original \texttt{.Rmd} file.
\end{enumerate}

\hypertarget{rubric}{%
\subsection{Rubric}\label{rubric}}

2 problems below will be graded according to,

\begin{itemize}
\tightlist
\item
  Correctness: For plots, the displays meet the required specifications.
  For conceptual questions, all parts are accurately addressed.
\item
  Attention to detail: Writing is clear and designs are elegant. For
  example, no superfluous marks are included, all axes are labeled, text
  is neither too small nor too large.
\item
  Code quality (if applicable): Code is concise but readable, properly
  formatted and commented.
\end{itemize}

The remaining problems will be graded for completeness.

\hypertarget{problems}{%
\subsection{Problems}\label{problems}}

\hypertarget{ikea-furniture}{%
\subsubsection{(1) Ikea Furniture}\label{ikea-furniture}}

The dataset below shows prices of pieces of Ikea furniture. We will
compare prices of different furniture categories and label the
(relatively few) articles which cannot be bought online.

\begin{Shaded}
\begin{Highlighting}[]
\KeywordTok{library}\NormalTok{(}\StringTok{"readr"}\NormalTok{)}
\NormalTok{ikea <-}\StringTok{ }\KeywordTok{read_csv}\NormalTok{(}\StringTok{"https://uwmadison.box.com/shared/static/iat31h1wjg7abhd2889cput7k264bdzd.csv"}\NormalTok{)}
\end{Highlighting}
\end{Shaded}

\begin{enumerate}
\def\labelenumi{\alph{enumi}.}
\item
  Make a plot that shows the relationship between the \texttt{category}
  of furniture and \texttt{price} (on a log-scale). Show each
  \texttt{item\_id} as a point -- do not aggregate to boxplots or
  ridgelines -- but make sure to jitter and adjust the size the points
  to reduce the amount of overlap. \emph{Hint: use the
  \texttt{geom\_jitter} layer.}
\item
  Modify the plot in (a) so that categories are sorted from those with
  highest to lowest average prices.
\item
  Color points according to whether they can be purchased online. If
  they cannot be purchased online, add a text label giving the name of
  that item of furniture. An example result is given by Figure 1.
\end{enumerate}

\hypertarget{penguins}{%
\subsubsection{(2) Penguins}\label{penguins}}

The data below measures properties of various antarctic penguins.

\begin{Shaded}
\begin{Highlighting}[]
\NormalTok{penguins <-}\StringTok{ }\KeywordTok{read_csv}\NormalTok{(}\StringTok{"https://uwmadison.box.com/shared/static/ijh7iipc9ect1jf0z8qa2n3j7dgem1gh.csv"}\NormalTok{)}
\end{Highlighting}
\end{Shaded}

Using either vega-lite or ggplot2, create a single plot that makes it
easy to answer both of these questions,

\begin{enumerate}
\def\labelenumi{\roman{enumi})}
\tightlist
\item
  How is bill length related to bill depth within and across species?
\item
  On which islands are which species found?
\end{enumerate}

(Notice that the answer to part (i) is an example of Simpson's paradox!)

\hypertarget{london-olympics}{%
\subsubsection{(3) 2012 London Olympics}\label{london-olympics}}

This exercise is similar to the Ikea furniture one, except that it will
be interactive. The data at this
\href{https://uwmadison.box.com/s/rzw8h2x6dp5693gdbpgxaf2koqijo12l}{link}
describes all participants in the London 2012 Olympics. From an
observable notebook, the following code can be used to derive a new
variable with a jittered Age variable, which will be useful in part (a).

\begin{Shaded}
\begin{Highlighting}[]
\ImportTok{import} \OperatorTok{\{}\NormalTok{ vl }\OperatorTok{\}} \ImportTok{from} \StringTok{"@vega/vega-lite-api"}
\ImportTok{import} \OperatorTok{\{}\NormalTok{ aq}\OperatorTok{,}\NormalTok{ op }\OperatorTok{\}} \ImportTok{from} \StringTok{"@uwdata/arquero"}
\NormalTok{data_raw }\OperatorTok{=} \VariableTok{aq}\NormalTok{.}\AttributeTok{fromCSV}\NormalTok{(}\ControlFlowTok{await} \AttributeTok{FileAttachment}\NormalTok{(}\StringTok{"All London 2012 athletes - ALL ATHLETES.csv"}\NormalTok{).}\AttributeTok{text}\NormalTok{())}
\NormalTok{data }\OperatorTok{=} \VariableTok{data_raw}\NormalTok{.}\AttributeTok{derive}\NormalTok{(}\OperatorTok{\{}\DataTypeTok{Age_}\OperatorTok{:}\NormalTok{ d }\KeywordTok{=>} \VariableTok{d}\NormalTok{.}\AttributeTok{Age} \OperatorTok{+} \FloatTok{0.25} \OperatorTok{*} \VariableTok{Math}\NormalTok{.}\AttributeTok{random}\NormalTok{() }\OperatorTok{\}}\NormalTok{)}
\end{Highlighting}
\end{Shaded}

\begin{enumerate}
\def\labelenumi{\alph{enumi}.}
\item
  Create a layered display that shows (i) the ages of athletes across
  sports and (ii) the average age within each sport. Use different marks
  for participants and for averages. To avoid overplotting, use the
  jittered \texttt{Age\_} variable defined in the code block above.
\item
  Sort the sports from lowest to highest average age. Add a tooltip so
  that hovering over an athlete shows their name. Your results should
  look something like the display in Figure \ref{fig:3}.
\end{enumerate}

\begin{figure}
  \centering
  \includegraphics[width=0.8\textwidth]{/Users/kris/Desktop/olympics.png}
  \caption{Example vega-lite result for Problem (3). Hovering over a tick mark
  shows the name of the athlete.}
  \label{fig:3}
\end{figure}

\hypertarget{traffic}{%
\subsubsection{(4) Traffic}\label{traffic}}

In lecture, we looked at the \texttt{geom\_density\_ridges} function. In
this exercise, we will instead use \texttt{geom\_ridgeline}, which is
useful whenever the heights of the ridges have been computed in advance.
We will use the traffic data read in below.

\begin{Shaded}
\begin{Highlighting}[]
\NormalTok{traffic <-}\StringTok{ }\KeywordTok{read_csv}\NormalTok{(}\StringTok{"https://uwmadison.box.com/shared/static/x0mp3rhhic78vufsxtgrwencchmghbdf.csv"}\NormalTok{)}
\end{Highlighting}
\end{Shaded}

Each row is a timepoint of traffic within a city in Germany. Using
\texttt{geom\_ridges}, make a plot of traffic over time, within each of
the cities. An example result is shown below.

\hypertarget{language-learning}{%
\subsubsection{(5) Language Learning}\label{language-learning}}

This problem will look at a simplified version of the data from the
study \emph{A critical period for second language acquisition: Evidence
from 2/3 million English speakers}, which measured the effect of the the
age of initial language learning on performance in grammar quizzes. We
have downloaded the raw data from the supplementary material and reduced
it down to the average and standard deviations of test scores within
(initial learning age) \(\times\) (current age-group) combinations. We
have kept a column \texttt{n} showing how many participants were used to
compute the associated statistics. The resulting data are available
\href{https://uwmadison.box.com/shared/static/m53dea9w5ipczs3d7nqdnxqxx5ao501b.csv}{here}.

\begin{enumerate}
\def\labelenumi{\alph{enumi})}
\tightlist
\item
  Using the \texttt{.derive()} command in arquero, create two new
  fields, \texttt{low} and \texttt{high}, giving confidence intervals
  for the means in each row. That is, derive new variables according to
  \(\hat{x} \pm 2 * \frac{1}{\sqrt{n}}\hat{\sigma}\).
\item
  Create a \texttt{markArea}-based ribbon plot showing confidence
  intervals for average test scores as a function of starting age.
  Include a line for the average score within that combination. An
  example result is shown in Figure \ref{fig:5}. Interpret the results
  of the study.
\end{enumerate}

\begin{figure}
  \centering
  \includegraphics[width=0.6\textwidth]{/Users/kris/Desktop/languages.png}
  \caption{Example result for problem (5).}
  \label{fig:5}
\end{figure}

\hypertarget{deconstruction}{%
\subsubsection{(6) Deconstruction}\label{deconstruction}}

Take a static screenshot from any of the visualizations in this
\href{https://www.theguardian.com/us-news/ng-interactive/2017/dec/20/bussed-out-america-moves-homeless-people-country-study}{article},
and deconstruct its associated visual encodings.

\begin{enumerate}
\def\labelenumi{\alph{enumi})}
\tightlist
\item
  What do you think was the underlying data behind the current view?
  What where the rows, and what were the columns?
\item
  What were the data types of each of the columns?
\item
  What encodings were used? How are properties of marks on the page
  derived from the underlying abstract data?
\item
  Is multi-view composition being used? If so, how?
\end{enumerate}

\end{document}
